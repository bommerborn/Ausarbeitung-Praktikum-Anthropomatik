\section{Features}

\subsection{Physikalische Größen}
Beschleunigung in X- und Y-Richtung zum Fließband.

Geschwindigkeit in X- und Y-Richtung zum Fließband.

X und Y Position auf dem Fließband.

Die Geschwindigkeit \( v = \frac{s}{t} \) und die Beschleunigung \( \Delta a = \frac{\Delta v}{\Delta t} \)
sind wichtige Merkmale. 

physikalische Eigenschaft Testobjekten -> Verhalten auf dem Band

Zylinder haben eine hohe Eigenbewegung. Somit weicht die Bewegung solcher Objekte stark von der Bandbewegung ab. Bei einem längeren Fließband konvergiert die Eigenbewegung der Objekte langsam mit der Bandbewegung. 

Quader bleiben auf dem Fließband liegen da die Objektbewegung langsam konvergiert. 

Kugeln erreichen das Ende des Bandes, das sich in die gleiche Richtung wie das Band bewegt. 

Die Beschleunigung wirkt auf die Objekte

\subsection{Einzelne Features}
Physikalische Größen am Anfang und am Ende des Fließbandes.





\subsection{Statistische Features}
Durchschnittswert der Beschleunigung, der Beschleunigungsänderung,  

Summe des Absolutbetrages von Position, Beschleunigung, Beschleunigungsänderung
